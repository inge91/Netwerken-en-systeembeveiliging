%%%%%%%%%%%%%%%%%%%%%%%%%%%%%%%%%%%%%%%%%

% Simple Sectioned Essay Template
% LaTeX Template
%
% This template has been downloaded from:
% http://www.latextemplates.com
%
% Note:
% The \lipsum[#] commands throughout this template generate dummy text
% to fill the template out. These commands should all be removed when 
% writing essay content.
%
%%%%%%%%%%%%%%%%%%%%%%%%%%%%%%%%%%%%%%%%%

%----------------------------------------------------------------------------------------
%	PACKAGES AND OTHER DOCUMENT CONFIGURATIONS
%----------------------------------------------------------------------------------------

\documentclass[12pt]{article} % Default font size is 12pt, it can be changed here

%\usepackage{geometry} % Required to change the page size to A4
\usepackage{subfigure}
\usepackage{placeins}
\usepackage{placeins}
\usepackage{hyperref}
%\geometry{a4paper}
\usepackage[margin=0.5in]{geometry}

\usepackage{graphicx} % Required for including pictures

\usepackage{float} % Allows putting an [H] in \begin{figure} to specify the exact location of the figure
\usepackage{amssymb} % Allows putting an [H] in \begin{figure} to specify the exact location of the figure
\usepackage{wrapfig} % Allows in-line images such as the example fish picture

\usepackage{lipsum} % Used for inserting dummy 'Lorem ipsum' text into the template

\linespread{1.2} % Line spacing

%\setlength\parindent{0pt} % Uncomment to remove all indentation from paragraphs

\graphicspath{{./Pictures/}} % Specifies the directory where pictures are stored

\begin{document}
\title{Netwerken en systeembeveiliging\\ Assignment 2}
\author{Inge Becht}
\date{\today}
\maketitle
\begin{verbatim}
Task 1
1.ip from request: 192.168.2.10 
  ip from destination: 192.16.191.44
2. 2 packets. This because in the end twice a reply is sent by the destination
host, and every time twice each intermediate host replies twice with an "TTL
exceeded" message.
3. The computer sends out a ping request with Time To Live 1. When this message
    reaches a host (which is not the desired destination)
    the time to live decreases to 0 and the reached host sets out a
    "TTL exceeded message" which means the destination is not yet reached, and so
    another message needs to be send with an incremented Time To Live by 1. Now 
    the request can make an extra hop towards the destination, but will be 
    exceeded again if this second host is not yet the destination. 
    This goes on until the destination is reached and a reply is sent.
    Each time a host sends back a "TTL exceeded" message the computer can
    calculate the length of the path so as to know the time to reach each hop.
4.  8 hops away in case hops consists of  intermediate hosts and destination
    host.
5. The destination host is 147.102.222.213
6. UDP works the same way as ICMP but instead of sending a ping request it gives
    the possibility of choosing a source and destination port for the request.
7. After 19 hops the message is received from the destination host that 
    the destination is not reachable. 

Task 2
8. pingable:
        www.facebook.com
        www.cwi.nl
        www.ntua.gr
        www.twitter.com 
    Only www.mit.edu doens't respond. It could be the server does not respond to
    ping requests, or a proxy blocks the request.
9.
 The last responding nodes is OC11-RTR-1-BACKBONE-2.MIT.EDU (18.168.1.41) 

10. for facebook:
        12.8ms
        13.2ms
        12.8ms
    mean time 12.76 ms

    for ntua.gr
        51.7
        51.5
        51.5
    mean time 51.56 ms

    for cwi.nl:
        0.855 ms
        0.743 ms
        0.731 ms
    mean time 0.77 ms
    
    There is quite a lot of difference between time for each host. 
    This could be explained by that the amount of hops for the ntua.gr is larger
    than the other two hosts, and thus more queueing delay could happen. This
    idea seems correct if you look at the nature of each host. ntua.gr takes
    longest as it is situated in greece, and facebook takes not as long because
    multiple ip adresses are available.
11.
    For www.twitter.com a jump between 7 ms and 82 ms happen at:
    xe-5-0-0.mpr1.lhr2.uk.above.net (64.125.24.77)  7.418 ms 
    xe-4-3-0.cr2.dca2.us.above.net (64.125.24.41)  82.134 ms 

    For www.mit.edu it is:
    ae-48-48.ebr2.London1.Level3.net (4.69.143.82)  8.181 ms 
    ae-44-44.ebr1.NewYork1.Level3.net (4.69.137.78)  76.738 ms 

    So the increase starts when the router crosses from the UK to the US.
    The jump in distance is quite big at that point until the next router which
    explains the sudden time increase.

12. The data:
    ping -c 10 -s 8000 www.cwi.nl
    PING www.cwi.nl (192.16.191.44) 8000(8028) bytes of data.
    8008 bytes from proxy2.cwi.nl (192.16.191.44): icmp_req=1 ttl=60 time=18.8 ms
    8008 bytes from proxy2.cwi.nl (192.16.191.44): icmp_req=2 ttl=60 time=20.5 ms
    8008 bytes from proxy2.cwi.nl (192.16.191.44): icmp_req=3 ttl=60 time=17.2 ms
    8008 bytes from proxy2.cwi.nl (192.16.191.44): icmp_req=4 ttl=60 time=17.6 ms
    8008 bytes from proxy2.cwi.nl (192.16.191.44): icmp_req=5 ttl=60 time=20.7 ms
    8008 bytes from proxy2.cwi.nl (192.16.191.44): icmp_req=6 ttl=60 time=26.4 ms
    8008 bytes from proxy2.cwi.nl (192.16.191.44): icmp_req=7 ttl=60 time=18.7 ms
    8008 bytes from proxy2.cwi.nl (192.16.191.44): icmp_req=8 ttl=60 time=16.5 ms
    8008 bytes from proxy2.cwi.nl (192.16.191.44): icmp_req=9 ttl=60 time=20.6 ms
    8008 bytes from proxy2.cwi.nl (192.16.191.44): icmp_req=10 ttl=60 time=53.0 ms

    --- www.cwi.nl ping statistics ---
    10 packets transmitted, 10 received, 0% packet loss, time 9013ms
    rtt min/avg/max/mdev = 2.749/10.253/50.538/14.402 ms

    ping -c 10 -s 8000 www.cwi.nl
    PING www.cwi.nl (192.16.191.44) 8000(8028) bytes of data.
    8008 bytes from proxy2.cwi.nl (192.16.191.44): icmp_req=1 ttl=60 time=18.8 ms
    8008 bytes from proxy2.cwi.nl (192.16.191.44): icmp_req=2 ttl=60 time=20.5 ms
    8008 bytes from proxy2.cwi.nl (192.16.191.44): icmp_req=3 ttl=60 time=17.2 ms
    8008 bytes from proxy2.cwi.nl (192.16.191.44): icmp_req=4 ttl=60 time=17.6 ms
    8008 bytes from proxy2.cwi.nl (192.16.191.44): icmp_req=5 ttl=60 time=20.7 ms
    8008 bytes from proxy2.cwi.nl (192.16.191.44): icmp_req=6 ttl=60 time=26.4 ms
    8008 bytes from proxy2.cwi.nl (192.16.191.44): icmp_req=7 ttl=60 time=18.7 ms
    8008 bytes from proxy2.cwi.nl (192.16.191.44): icmp_req=8 ttl=60 time=16.5 ms
    8008 bytes from proxy2.cwi.nl (192.16.191.44): icmp_req=9 ttl=60 time=20.6 ms
    8008 bytes from proxy2.cwi.nl (192.16.191.44): icmp_req=10 ttl=60 time=53.0 ms

    --- www.cwi.nl ping statistics ---
    10 packets transmitted, 10 received, 0% packet loss, time 9013ms
    rtt min/avg/max/mdev = 16.599/23.057/53.054/10.341 ms

    In both cases there was no packet loss, but the average rtt difference is 
    10 ms. The propagation time for the second set of packets was higher.

task 3

13. The output for the traceroute from the Greece host to surfnet:
   traceroute to www.surfnet.nl (145.0.2.10), 15 hops max, 52 byte packets
    1  * * *
    2  grnetRouter.ntua-primary.koletti-1.access-link.grnet.gr 
            (194.177.209.117)  0.378 ms  0.369 ms  0.382 ms
    3  eie2-to-kol1.backbone.grnet.gr (195.251.27.54)  0.471 ms  0.579 ms  0.463 ms
    4  grnet.rt1.ath2.gr.geant.net (62.40.124.89)  0.525 ms  0.531 ms  0.538 ms
    5  so-4-2-0.rt1.mil.it.geant2.net (62.40.112.153)  28.986 ms  29.013 ms  28.994 ms
    6  as0.rt1.gen.ch.geant2.net (62.40.112.33)  36.396 ms  36.437 ms  36.406 ms
    7  so-4-0-0.rt1.fra.de.geant2.net (62.40.112.161)  44.194 ms  44.174 ms  44.188 ms
    8  so-4-0-0.rt1.ams.nl.geant2.net (62.40.112.10)  51.194 ms  51.190 ms  51.154 ms
    9  surfnet-gw.rt1.ams.nl.geant.net (62.40.124.158)  51.291 ms  51.295 ms  59.459 ms
   10  AE2.500.JNR01.Asd001A.surf.net (145.145.80.78)  51.237 ms  51.222 ms  51.218 ms
   11  V1131.sw4.amsterdam1.surf.net (145.145.19.170)  51.450 ms  51.392 ms  51.408 ms
   12  www.surfnet.nl (145.0.2.10)  51.412 ms !Z  51.400 ms !Z  51.387 ms !Z
    
    The traceroute link from Switzerland:
   traceroute to www.surfnet.nl (145.0.2.10), 30 hops max, 60 byte packets
    1  swiCS5-V108.switch.ch (130.59.108.5)  0.306 ms  0.358 ms  0.430 ms
    2  swiZH2-10GE-3-1.switch.ch (130.59.36.138)  0.410 ms  0.503 ms  0.594 ms
    3  swiCE3-10GE-3-1.switch.ch (130.59.36.1)  36.782 ms * *
    4  switch-bckp.rt1.par.fr.geant.net (62.40.124.81)  12.878 ms  12.881 ms  12.873 ms
    5  as0.rt1.lon.uk.geant2.net (62.40.112.106)  20.365 ms  20.423 ms  20.465 ms
    6  as1.rt1.ams.nl.geant2.net (62.40.112.137)  28.365 ms  28.392 ms  28.360 ms
    7  surfnet-gw.rt1.ams.nl.geant.net (62.40.124.158)  31.192 ms  31.230 ms  31.275 ms
    8  AE2.500.JNR01.Asd001A.surf.net (145.145.80.78)  28.461 ms  28.482 ms  28.511 ms
    9  V1131.sw4.amsterdam1.surf.net (145.145.19.170)  28.629 ms  28.589 ms  28.640 ms
   10  www.surfnet.nl (145.0.2.10)  28.577 ms !X  28.528 ms !X  28.512 ms !X

    number of routes for the greek host: 1 route with 12 hops
    number of Switzerland host: 1 route with 10 hops

    average round trip delay to destination for Greece: 416.386 ms
    average round trip delay to destination for Switzerland:216.19 ms
        
14. There are 3 links the same (4 in case you include de destination):
      surfnet-gw.rt1.ams.nl.geant.net (62.40.124.158)  
      AE2.500.JNR01.Asd001A.surf.net (145.145.80.78)
      V1131.sw4.amsterdam1.surf.net (145.145.19.170)
15. The biggest delay for the Greek host happens at the first contact with
    an Italian server. This is probably an important server spot that
    handles a lot of packets.
    The biggest delay for the Switzerland host happens at the first contact
    with ip 130.59.36.1. This is probably more because of a temporary
    malfunctioning server (as only 1 connection could be made instead of 3).
16. trace output from Australia to www.google.com:
     1  gigabitethernet3-3.exi1.melbourne.telstra.net (203.50.77.49)  0.440 ms  0.223 ms  0.245 ms
     2  bundle-ether3.exi-core1.melbourne.telstra.net (203.50.80.1)  0.867 ms  2.362 ms  2.995 ms
     3  bundle-ether12.chw-core2.sydney.telstra.net (203.50.11.74)  18.608 ms  23.474 ms  23.982 ms
     4  bundle-ether1.chw48.sydney.telstra.net (203.50.6.154)  25.605 ms  23.476 ms  23.981 ms
     5  74.125.50.1 (74.125.50.1)  15.111 ms  15.105 ms  15.111 ms
     6  66.249.95.226 (66.249.95.226)  15.359 ms  15.355 ms  15.362 ms
     7  72.14.237.53 (72.14.237.53)  16.611 ms  16.603 ms  16.610 ms
     8  syd01s05-in-f19.1e100.net (74.125.237.51)  16.112 ms  16.104 ms  16.112 ms
   
    trace output from Switzerland to www.google.com:
    traceroute to www.google.com (173.194.44.208), 30 hops max, 60 byte packets
     1  swiCS5-V108.switch.ch (130.59.108.5)  0.423 ms  0.507 ms  0.549 ms
     2  swiZH2-10GE-3-1.switch.ch (130.59.36.138)  4.292 ms  4.407 ms  4.494 ms
     3  swiIX1-10GE-3-3.switch.ch (130.59.36.129)  0.439 ms  0.437 ms  0.449 ms
     4  equinix-zurich.net.google.com (194.42.48.58)  62.291 ms  62.292 ms  62.284 ms
     5  209.85.243.127 (209.85.243.127)  1.183 ms  1.326 ms  1.469 ms
     6  173.194.44.208 (173.194.44.208)  0.720 ms  0.737 ms  0.724 ms
    
     There are no links the same. Google uses multiple ip addresses.
     Switzerland connects with Zurich and the Australian host ends up
     connecting to a router in Sydney
17. A packet is send 3 times by default when using traceroute. It seems that
    different routers were reached both times, which could be because of
    load-balancing.
\end{verbatim}

\end{document}

